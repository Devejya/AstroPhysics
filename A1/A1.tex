\documentclass{article}
\usepackage[utf8]{inputenc}
\usepackage{gensymb}
\usepackage{textcomp}
\usepackage{amsmath}
\usepackage{amssymb}
\usepackage{hyperref}
\hypersetup{
    colorlinks=true,
    linkcolor=blue,
    filecolor=magenta,      
    urlcolor=cyan,
}
\usepackage{pythonhighlight}
\usepackage[utf8]{inputenc}
\usepackage{graphicx}
\DeclareUnicodeCharacter{2212}{-}
\title{PHYS270: Assignment 1}
\author{Devejya Raghuvanshi }
\date{Winter 2021}

\begin{document}

\maketitle

\section*{Problem 1}
    \subsection*{Part a.}
        Using the law of cosines, the relation between Altitude, Declination, Latitude and Hour Angle is the following:
        \[sin A = sin(\delta)sin(\beta) + cos(\delta)cos(\beta)cos(HA)\]
        Where, $A:=$ Altitude, $\delta:=$ Declination, $\beta:=$ Latitude, $HA:=$ Hour Angle.
        Since, $\beta$ is constant, therefore, $HA$ and $\delta$ are the variables here. To maximise $A$, we need to minimise $cos(\delta)cos(\beta)cos(HA)$. Thus, leading to $cos(HA) = 1 \Rightarrow HA = 0$. \newline
        
        In addition, to minimise $sin(A)$ wrt. $\delta$, $\frac{dsin(A)}{d\delta} = 0$.
        
        \[\Rightarrow cos(\delta)sin(\beta) - sin(\delta)cos(\beta)\cdot 1 = 0\]
        \[\Rightarrow cos(\delta)sin(\beta) = sin(\delta)cos(\beta)\]
        \[\Rightarrow tan(\beta) = tan(\delta)\]
        \[\beta = \delta\]
        
    \subsection*{Part b.}
        \subsubsection*{i.}RA, Dec in J2000 for:
            \begin{enumerate}
                \item Alpha Centauri: (RA, Dec) = (14h 39m 37s, $-60\degree$ 50' 2")
                \item GN-z11: (RA, Dec) = (12h 36m 25.46s, $+62\degree$ 14' 31.4")
                \item Sag $A^*$: (RA, Dec) = (17h 45m 40.0409s, $−29\degree$ 0' 28.118")
                \item 3C273: (RA, Dec) = (12h 29m 7s, $+2\degree$ 3' 9")
            \end{enumerate}
        \subsubsection*{ii.}
        Best observatory for the above celestial objects would be:
            \begin{enumerate}
                \item Alpha Centauri: Since, GS is located 30$\degree$ South and Alpha Centauri is at $\delta -60\degree$, $\therefore$ GS is the better observatory for observing Alpha Centauri.
                \item GN-z11: Since, GN is located 19$\degree$ North and GN-z11 is at $\delta ~ +62\degree$, $\therefore$ GN is the better observatory for observing GN-z11.
                \item Sag $A^{*}$: Since, GS is located 30$\degree$ South and Sag $A^{*}$ is at $\delta ~ -29\degree$, $\therefore$ GS is the better observatory for observing Sag $A^{*}$.
                \item 3C273: Since, GN is located 19$\degree$ North and 3C273 is at $\delta ~ +2\degree$, $\therefore$ GN is the better observatory for observing 3C273.
            \end{enumerate}
            
        \subsubsection*{iii.}
            At Vernal Equinox, Sun's $\alpha = 0$. $\therefore$ At midnight, Sun's $\alpha = 12$ hours.
            Since, the Sun will be at the same location next midnight in 24 hours, $\therefore$ on the next midnight, Sun's $\alpha$ will be $24\times \frac{60}{365}$ minutes + $12 hours$.\newline
            Thus, to find the date on which the object will transit at midnight, we need to find the number of days it will take for the Sun's $\alpha$ will be equal to the Celestial Objects RA. (The code \ref{code: 1)iii)} can be found in Appendix)\newline
            
            Number of Days (Alpha Centauri):  40\newline
            \textbf{Date on midnight (Alpha Centauri) 2000-04-29} \newline
            Number of Days (GN-z11):  9\newline
            \textbf{Date on midnight (GN-z11) 2000-03-29} \newline
            Number of Days (SagA*):  87\newline
            \textbf{Date on midnight (SagA*) 2000-06-15} \newline
            Number of Days (3C273):  7\newline
            \textbf{Date on midnight (3C273) 2000-03-27}\newline
            
        \subsubsection*{iv.}
            Since, \[sin(A) = sin(\delta)sin(\beta) + cos(\delta)cos(\beta)cos(HA)\]
            at max altitude, HA = 0. $\therefore$ Using, \[sin(A) = sin(\delta)sin(\beta) + cos(\delta)cos(\beta)\]
            We can find the max altitude of each Celestial Object at the required Observatory using the above equation. (Code \ref{code: 1)iv)} in Appendix)
            The result is the following:
            \textbf{Max Altitude of Alpha Centauri at GS:  61.07} $\degree$\newline
            \textbf{ Max Altitude of GN-z11 at GN:  47.58$\degree$}\newline
            \textbf{ Max Altitude of SagA* at GS:  88.75$\degree$}\newline
            \textbf{Max Altitude of 3C273 at GN:  72.23$\degree$}\newline
            
        \subsubsection*{v.}
        
        To find the max time per night during which the object remains visible, at more than 30 degrees from the horizon, we need to find the HA when the object rises or sets at Altitude = 30 $\degree$.
        Using,
        \[HA = \arccos[{\frac{\sin{A}}{\cos{\delta}cos{\beta}} - \tan{\delta}\tan{\beta}}]\]
        
        Therefore, the visibility will be min(2$\times$HA, number of hours of darkness). (Code \ref{code: 1)v)} in Appendix.)
        
        The results are the following:
        
        \textbf{Alpha Centauri Visibility Hours:  12.21}\newline
        \textbf{GN-z11 Visibility Hours:  10.96}\newline
        \textbf{Sag-A* Visibility Hours:  12.92}\newline
        \textbf{3C273 Visibility Hours:  11.0}\newline
        
    \subsubsection*{Final Table:}
        The table \ref{table:1} summarizes the results from question 1b.
        
        \begin{table}[h!]
        \centering
        \begin{tabular}{|c| c| c| c| c | c | c|} 
         \hline
         Target & RA & Dec & Observatory & Date & Max A. & Obs. Time (hours) \\
         \hline
         \hline
         Alpha Centauri & 14h 39m 37s & -60$\degree$50'02" & GS & April 29 & 61$\degree$ & 12.21 \\
         \hline
         GN-z11 & 12h 36m 25.46s & 62$\degree$14'31.4" & GN & March 29 & 48$\degree$ & 10.96 \\
         \hline
         Sag A* & 17h 45m 40.041s & -29$\degree$00'28.1" & GS & June 15 & 89$\degree$ & 12.92 \\
         \hline
         3C273 & 12h 29m 6.7s & +2$\degree$03'09" & GN & March 27 & 72$\degree$ & 11.0 \\
         \hline
        \end{tabular}
        \caption{Results from Question 1B.}
        \label{table:1}
        \end{table}
        
        
        
\section*{Problem 2}
    \subsection*{Part a)}
        \ref{fig:Path of Star}
        \begin{figure}[htp]
            \centering
            \includegraphics[width=5cm]{A1Q2_1.png}
            \caption{Path of Star}
            \label{fig:Path of Star}
        \end{figure}
        (Code \ref{code: 2)a)} in Appendix)
    \subsection*{Part b)}
        Proper Motion := Change in (RA, Dec) over a years time.
        
        Therefore, following the steps below, we can find the proper motion:
        \begin{itemize}
            \item Pick a sample of days a year apart from each other.
            \item Find the RA and Dec on those days for the star.
            \item Find the slope of RA vs. Days and Dec vs. Days
            \item Use the slopes, to find the proper motion, using:
            \[proper motion = \sqrt{(Slope_{RA})^2+(Slope_{Dec})^2}\times 365.25\]
        \end{itemize}
        (Code \ref{code: 2)b)} in Appendix)
        The resulting Proper Motion value is 349.78 mas/year
        
        Parallax := Maximum perpendicular distance between the RA vs. Dec linear fit and the position of the star.
        
        (Code \ref{code: 2)b)} in Appendix)
    \subsection{Part C)}
        Since d (in pc) = $\frac{1}{p (arcseconds)}$.
        $\Rightarrow d = 7.72 pc$.
        
        Therefore, proper motion = 1.31 $\times 10^{-5}$ pc/year = 12.8 km/s
        
        
\section*{Appendix}

Question 1) iii) 
\begin{equation}
   \label{code: 1)iii)} 
\end{equation}
\begin{python}
import math
import numpy as np 


def RA_to_hours(RA: list) -> float:
    '''Return hours given a list of hours, minutes and days.'''
    hours, minutes, seconds = RA[0], RA[1], RA[2]
    return hours + minutes/60 + seconds/(60*60)


def number_days_till_midnight(RA_hours: float) -> int:
    '''Return the number of days when object found at RA (in hours) is at midnight.'''
    number_of_days = 0
    if RA_hours == 12:
        return 0
    else:
        midnighthours = 12
        while midnighthours < RA_hours:
            midnighthours += 3.945/60
            number_of_days += 1
        return number_of_days-1

if __name__ == "__main__":
    from datetime import datetime  
    from datetime import timedelta 

    vernal_equinox_date = datetime.strptime("2000-03-20", "%Y-%m-%d")

    #Alpha Centauri
    object_hours_alpha = RA_to_hours([14, 39, 37])
    number_of_days_alpha = number_days_till_midnight(object_hours_alpha)
    print('Number of Days (Alpha Centauri): ', number_of_days_alpha)
    print("Date on midnight (Alpha Centauri)", vernal_equinox_date+timedelta(days=number_of_days_alpha))

    #GN-z11
    object_hours_gn = RA_to_hours([12, 36, 25.46])
    number_of_days_gn = number_days_till_midnight(object_hours_gn)
    print('Number of Days (GN-z11): ',number_of_days_gn)
    print("Date on midnight (GN-z11)", vernal_equinox_date+timedelta(days=number_of_days_gn))

    #SagA*
    object_hours_sag = RA_to_hours([17, 45, 40.041])
    number_of_days_sag = number_days_till_midnight(object_hours_sag)
    print('Number of Days (SagA*): ', number_of_days_sag)
    print("Date on midnight (SagA*)", vernal_equinox_date+timedelta(days=number_of_days_sag))

    #3C273
    object_hours_3C = RA_to_hours([12, 29, 6.7])
    number_of_days_3C = number_days_till_midnight(object_hours_3C)
    print('Number of Days (3C273): ',number_of_days_3C)
    print("Date on midnight (3C273)", vernal_equinox_date+timedelta(days=number_of_days_3C))

\end{python}
\smallbreak
Question 1) iv) 
\begin{equation}
   \label{code: 1)iv)} 
\end{equation}
\begin{python}
import math

def max_altitude(delta: list, observatory: str) -> float:
    ''' '''
    if observatory == "GN":
        beta = 19.8238
    if observatory == "GS":
        beta = -30.24075
    delta_radians = math.radians(delta[0] + delta[1]/60 + delta[2]/(60*60))
    beta_radians = math.radians(beta)
    return math.degrees(math.asin(math.sin(delta_radians)*math.sin(beta_radians) + math.cos(beta_radians)*math.cos(delta_radians)))


if __name__ == "__main__":
    #Alpha Centauri
    print ("Max Altitude of Alpha Centauri at GS: ", max_altitude([-60, 50, 2], "GS"))
    #GN-z11
    print ("Max Altitude of GN-z11 at GN: ", max_altitude([62, 14, 31.4], "GN"))
    #SagA*
    print ("Max Altitude of SagA* at GS: ", max_altitude([-29, 00, 28.1], "GS"))
    #3C273
    print ("Max Altitude of 3C273 at GN: ", max_altitude([+2, 3, 9], "GN"))
\end{python}

Question 1)v)
\begin{equation}
   \label{code: 1)v)} 
\end{equation}
\begin{python}
import math


def visibility_hours(A: float, delta: list, observatory: str, dark_hours: float) -> float:
    ''' '''
    if observatory == "GN":
        beta = 19.8238        
    if observatory == "GS":
        beta = -30.24075
    delta_radians = math.radians(delta[0] + delta[1]/60 + delta[2]/(60*60))
    beta_radians = math.radians(beta)
    HA = math.degrees(math.acos((math.sin(A)/(math.cos(delta_radians)*math.cos(beta_radians))) - 
        (math.tan(delta_radians)*math.tan(beta_radians))))

    return min(2*HA/6, dark_hours)


if __name__ == "__main__":
    A = math.radians(30)
    #Alpha Centauri
    alpha_dark_hours = 24-11.79
    print("Alpha Centauri Visibility Hours: ", visibility_hours(A, [-60, 50, 2], "GS", alpha_dark_hours))

    #GN-z11
    gn_dark_hours = 24-13.04
    print("GN-z11 Visibility Hours: ", visibility_hours(A, [62, 14, 31.4], "GN", gn_dark_hours))

    #SagA*
    sag_dark_hours = 24-11.08
    print("Sag-A* Visibility Hours: ", visibility_hours(A, [-29, 00, 28.1], "GS", sag_dark_hours))

    #3C273
    c_dark_hours = 24-13.00
    print("3C273 Visibility Hours: ", visibility_hours(A, [+2, 3, 9], "GN", c_dark_hours))
\end{python}

Question 2 a)
\begin{equation}
   \label{code: 2)a)} 
\end{equation}
\begin{python}
from astropy.io import ascii
import matplotlib.pyplot as plt


def dRAvsdDec(filename: str, columns) -> None:
    ''' Plot the deltaRA vs delta Dec given a file containing the position offset in
    milliarcseconds, of a star as a function of time'''
    data =ascii.read(filename, names=columns)
    fig,ax=plt.subplots()
    ax.scatter(data['RA'],data['Dec'])
    #plt.axis([-430,400,-550,280])
    plt.xlabel('$\Delta$RA (mas)')
    plt.ylabel('$\Delta$Dec (mas)')
    plt.show()
    
    return None
    
if __name__=="__main__":
    filename = 'phy270_ass1_parallax.txt'
    columns = ['Days', 'RA', 'Dec']
    dRAvsdDec(filename, columns)    
\end{python}

Question 2 b)
\begin{equation}
   \label{code: 2)b)} 
\end{equation}
\begin{python}

import datetime
import numpy as np
from astropy.io import ascii
from scipy.stats import linregress


def proper_motion(filename: str) -> float:
    data = ascii.read(filename, names=['Days', 'RA', 'Dec'])

    days_sample = np.arange(3)*365.25

    sampleRA = np.interp(days_sample,data['Days'],data['RA'])
    sampleDec = np.interp(days_sample,data['Days'],data['Dec'])

    ra = linregress(days_sample,sampleRA)
    dec = linregress(days_sample,sampleDec)

    slope_ra = ra[0]
    slope_dec = dec[0]

    proper_motion_value = np.sqrt(slope_ra**2+slope_dec**2)*365.25
    
    return proper_motion_value
    
    
def parallax(filename: str) -> float:

    data = ascii.read(filename, names=['Days', 'RA', 'Dec'])

    days_sample = np.arange(4)*365.25

    sampleRA = np.interp(days_sample,data['Days'],data['RA'])
    sampleDec = np.interp(days_sample,data['Days'],data['Dec'])

    ra = linregress(days_sample,sampleRA)
    dec = linregress(days_sample,sampleDec)

    # Fit a line to RA as a function of Dec for the annually-separated points.  Then calculaate 
    # the maximum perpendicular distance from this line to get the parallax	
    ra_v_dec =linregress(ra,dec)
    ra_v_dec_slope = ra_v_dec[0]
    ra_v_dec_intercept = ra_v_dec[1]

    x_min = (ra_v_dec_slope*(data['Dec']-ra_v_dec_intercept)+data['RA'])/(1+ra_v_dec_slope**2)
    y_min = x_min*ra_v_dec_slope + ra_v_dec_intercept

    return np.max(np.sqrt((x_min-data['RA'])*(x_min-data['RA']) + (y_min-data['Dec'])*(y_min-data['Dec'])))
\end{python}
\end{document}
